%% start of file `template-zh.tex'.
%% Copyright 2006-2012 Xavier Danaux (xdanaux@gmail.com).
%
% This work may be distributed and/or modified under the
% conditions of the LaTeX Project Public License version 1.3c,
% available at http://www.latex-project.org/lppl/.


\documentclass[12pt,a4paper,sans]{moderncv}   % possible options include font size ('10pt', '11pt' and '12pt'), paper size ('a4paper', 'letterpaper', 'a5paper', 'legalpaper', 'executivepaper' and 'landscape') and font family ('sans' and 'roman')

% moderncv 主题
\moderncvstyle{casual}                        % 选项参数是 ‘casual’, ‘classic’, ‘oldstyle’ 和 ’banking’
\moderncvcolor{blue}                          % 选项参数是 ‘blue’ (默认)、‘orange’、‘green’、‘red’、‘purple’ 和 ‘grey’
%\nopagenumbers{}                             % 消除注释以取消自动页码生成功能

% 字符编码
\usepackage[utf8]{inputenc}                   % 替换你正在使用的编码
\usepackage{CJKutf8}

% 调整页面出血
\usepackage[scale=0.85]{geometry}
%\geometry{left=1.5cm,right=1.5cm,top=1.0cm,bottom=1.5cm}
%\addtolength{\textheight}{5cm}
%\addtolength{\textwidth}{2cm}
%\setlength{\hintscolumnwidth}{3cm}           % 如果你希望改变日期栏的宽度

% 个人信息
\firstname{}
\familyname{Jiang Ji}
%\title{中科院计算所}                      % 可选项、如不需要可删除本行
%\address{中科院计算所}{}             % 可选项、如不需要可删除本行
\mobile{135-010-21029}                         % 可选项、如不需要可删除本行
%\phone{010-62601025}                          % 可选项、如不需要可删除本行
%\fax{+3~(456)~789~012}                            % 可选项、如不需要可删除本行
\email{jiangji.cn@gmail.com}                    % 可选项、如不需要可删除本行
%\homepage{www.xialongli.com}                  % 可选项、如不需要可删除本行
%\extrainfo{附加信息 (可选项)}                  % 可选项、如不需要可删除本行
%\photo[64pt][0.4pt]{picture}                  % ‘64pt’是图片必须压缩至的高度、‘0.4pt‘是图片边框的宽度 (如不需要可调节至0pt)、’picture‘ 是图片文件的名字;可选项、如不需要可删除本行
%\quote{引言(可选项)}                           % 可选项、如不需要可删除本行


% 显示索引号;仅用于在简历中使用了引言
%\makeatletter
%\renewcommand*{\bibliographyitemlabel}{\@biblabel{\arabic{enumiv}}}
%\makeatother

% 分类索引
%\usepackage{multibib}
%\newcites{book,misc}{{Books},{Others}}
%----------------------------------------------------------------------------------
%            内容
%----------------------------------------------------------------------------------
\begin{document}
\begin{CJK}{UTF8}{kai}                       % 详情参阅CJK文件包
\maketitle

\vspace{-1.4cm}
\parbox[t]{6cm}{\textbf{ICT, CAS}
}
\parbox[t]{7cm}{\textbf{Mobile:} 135-010-21029
}
\parbox[t]{6cm}{\textbf{E-mail:} jiangji.cn@gmail.com
}

\section{Objective:}
\cventry{}{Tencent - Backend Develop - Technology and Engineering Group}{}{}{}{}

\vspace{-0.3cm}
\section{Education}
%\cventry{年 -- 年}{学位}{院校}{城市}{\textit{成绩}}{说明}  % 第3到第6编码可留白
\cventry{2010--now}{Master Degree}{Computer Architecture}{Institute of Computing Technology, Chinese Academy of Science}{}{}
\cventry{2006--2010}{Bachelor Degree}{Institute of Microelectronics }{Tsinghua University}{}{}


%\section{毕业论文}
%\cvitem{题目}{\emph{题目}}
%\cvitem{导师}{导师}
%\cvitem{说明}{\small 论文简介}
%\vspace{-0.3cm}
\section{Experience}
\cventry{2012/5 -- now}{A study of SSD-based multi-level cache}{Designer, main Implementor}{}{}{
\begin{itemize}%
\item {\normalsize As for SSD's unique characters such as high OPS/\$ and asymmetric-write, we conduct a deep analysis and experiment on different combination of cache replacement and prefetch policies and propose a new SSD-aware cache management algorithm, effectively reducing execution time }
\item \normalsize In charge of procedure design, implement and debug cache simulator and a number of cache algorithms(LRU, LIRS, MQ, Linux, AMP, etc.), evaluate and analyse the system with several real-world workload.
%\item {\normalsize 针对SSD性价比高及读写性能失衡等特点,通过分析与测试多种缓存替换算法与预取算法管理异构缓存系统(DRAM与SSD)的结果与原因,提出为SSD设计缓存替换算法的原则并设计实现了新的缓存管理算法,减少SSD 的写入量同时提高命中率,有效减少执行时间.}
%\item \normalsize 负责多级异构缓存模拟器的详细设计,实现并调试缓存模拟器及多种缓存替换与预取算法(包括LRU,LIRS,MQ,Linux,AMP等),使用MSR等多种负载进行测试与分析.
\end{itemize}}

\vspace{0.2cm}

\cventry{2011/10 -- 2012/5}{Virtual machine image management system}{main Designer, co-implementor}{}{}{
\begin{itemize}%
\item \normalsize In order to meet a more rigid demand for virtual disk image safety and system stability, based on a thorough survey, we choose OCFS2 against GFS2 as a prototype system and develop a virtual machine image management system.
\item \normalsize In charge of OCFS2's source code survey and conduct a evaluation of performance and stability, design and implement a "virtual disk image lock" module to protect disk image, co-design and implement a new high availability module upon origin(including error detect, quorum, fence, etc.)
\item \normalsize  Key points including: 1) add an additional lock based on Distributed Lock Manager(DLM) and check status on critical path. 2) re-design keep-alive state machine to combine disk heartbeat and network heartbeat in order to improve error detecting efficiency.
%\item \normalsize 针对虚拟化生产环境中对磁盘镜像安全性与系统稳定性的特殊需求,通过综合调研测试,在OCFS2与GFS2中选定OCFS2作为原型开发虚拟机镜像管理系统;
%\item \normalsize 负责前期对OCFS2源代码分析调研并对性能及稳定性进行测试;负责详细设计并实现了虚拟磁盘锁以达到保护虚拟机磁盘镜像;参与设计及调试故障检测与恢复等可靠性模块的改进;关键技术包括:1)基于分布式锁管理(DLM)模块添加新锁并在关键路径进行检查 2)重新设计高可用(Keepalive)状态机,结合磁盘心跳与网络心跳,提高故障检测效率
\end{itemize}}

\vspace{0.2cm}

\cventry{2011/4 -- 2011/11}{SSD-based HDFS metadata management system}{Designer, main Implementor}{}{}{
\begin{itemize}%
\item \normalsize In order to mitigate several defects of HDFS's all-in-memory tree-like metadata management architecture(such as poor scalability), we build a hash-based hybrid-storage-aware metadata management module that can improve scalability and reduce recovery time while sacrifice little peak performance.
\item \normalsize In charge of designing and implementing metadata management module. Key points including: 1) Store filename's fingerprint to achieve low per key memory overhead. 2) Reducing hash collision by actions like combining two hash function, storing parent ino, ect. Effectively alleviate SSD's read amplification. 3) Use a metadata-box and copy-on-write way to store metadata, guarantee file system's consistency and make a better use of SSD.
%\item \normalsize 针对Hadoop中HDFS采用树型全内存结构管理文件系统元数据的不足(元数据容量受限,可扩展性差),使用哈希索引的方式实现了基于DRAM与SSD混合存储的元数据管理模块, 以略微降低峰值性能为代价大大提高系统的可扩展性,同时减少系统的故障恢复时间;
%\item \normalsize 负责详细设计与实现元数据管理模块,关键技术包括:1)通过在内存中存储文件名指纹等方式压缩单条记录内存开销 2)通过拼装哈希值、存储父节点号等措施减少哈希冲突,有效减少SSD读放大效应 3)以元数据盒方式管理元数据,采用COW方式更新,保证一致性同时降低锁开销并提高SSD利用效率
\end{itemize}}

%\vspace{-0.3cm}

\section{Skill}
\cvitem{Programming}{Skilled in C/C++ programming in Linux environment, familiar with and experienced in programming and debugging within Linux kernel, enterprise project experience, Deep understanding of local/distributed file system}
\cvitem{Algorithm}{Strong background of algorithm and data structure, deep understanding of HashTable、Bloomfilter, etc.}
\cvitem{Tools}{Skilled in gdb, Shell programming, Vim, Gnuplot,AWK, \LaTeX{}, MS Office}
\cvitem{Language}{Passed CET4/6((600,560)/710),TOEFL(107/120),GRE(1380/1600), capable of oral/written communication in English}

%\vspace{-0.3cm}
\section{Activities}
\cvlistitem{Participated in Charity Association of Tsinghua University and acted as vice-chairman, participated/organized several times of charity activities such as "Graduation Donation"}
\cvlistitem{Fascinated in sports such as basketball, badminton, ping-pong, football, selected in class team respectively}
%\cvlistitem{喜欢读书,尤其历史、武侠、科幻等题材}


%\renewcommand{\listitemsymbol}{-}             % 改变列表符号
%
%\section{其他 2}
%\cvlistdoubleitem{项目 1}{项目 4}
%\cvlistdoubleitem{项目 2}{项目 5\cite{book1}}
%\cvlistdoubleitem{项目 3}{}

% 来自BibTeX文件但不使用multibib包的出版物
%\renewcommand*{\bibliographyitemlabel}{\@biblabel{\arabic{enumiv}}}% BibTeX的数字标签
%\nocite{*}
%\bibliographystyle{plain}
%\bibliography{publications}                    % 'publications' 是BibTeX文件的文件名

% 来自BibTeX文件并使用multibib包的出版物
%\section{出版物}
%\nocitebook{book1,book2}
%\bibliographystylebook{plain}
%\bibliographybook{publications}               % 'publications' 是BibTeX文件的文件名
%\nocitemisc{misc1,misc2,misc3}
%\bibliographystylemisc{plain}
%\bibliographymisc{publications}               % 'publications' 是BibTeX文件的文件名

\clearpage\end{CJK}
\end{document}


%% 文件结尾 `template-zh.tex'.
